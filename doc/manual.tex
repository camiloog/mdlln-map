\documentclass[11pt,letterpaper]{article}
\usepackage[utf8]{inputenc}
\usepackage[spanish]{babel}
\usepackage{csquotes} % recomendation of biblatex alongside babel
\usepackage{amsmath}
\usepackage{amsfonts}
\usepackage{amssymb}
\usepackage{graphicx}
\usepackage[left=2cm,right=2cm,top=2cm,bottom=2cm]{geometry}
\usepackage{hyperref}
\hypersetup
{
    colorlinks=true,
    citecolor=blue,
    linkcolor=blue,
    filecolor=blue,      
    urlcolor=blue,
    pdfstartview={FitH}
}

\author
{
	Camilo Ocampo\\
 	{\tt miloog@gmail.com}
}
\title{\bf MAPA INTERACTIVO PARA LA IDENTIFICACIÓN DE LAS PROBLEMÁTICAS AMBIENTALES EN EL MUNICIPIO DE MEDELLÍN}
\date{}

\begin{document}
\maketitle
\section{Introducción}
El mapa interactivo aquí documentado representa un producto enfocado y minimo viable para mostrar información relevante de las problemáticas ambientales en el municipio de Medellín. Se seleccionó el desarrollo web como plataforma para el aplicativo dada su versatilidad a la hora de compartir la información tanto {\it online} como {\it offline}. Para su desarrollo se adoptó la filosofía {\it mobil first} que orienta el diseño hacia una plataforma responsive. Las herramientas fundamentales usadas para el desarrollo fueron HTML5, CSS3 y Javascript. Con el objetivo de mejorar la velocidad y eficiencia en el desarrollo se incluyeron dentro del entorno de desarrollo: El sistema de control de versiones Git \cite{git}, los administradores de paquetes npm \cite{npm} y Bower \cite{bower}, y el ejecutador de tareas Gulp \cite{gulp}. Finalmente se apoyó el diseño e interactividad de la plataforma haciendo uso de los {\it frameworks}: Bootstrap \cite{bootstrap}, jQuery y Leaflet \cite{leaflet}.

\bibliographystyle{unsrt}
\bibliography{references}

\end{document}